\documentclass[a4paper,12pt]{book}
\usepackage[utf8]{inputenc}
\usepackage[T1]{fontenc}
\usepackage[brazilian]{babel}
\usepackage{graphicx}
\usepackage{hyperref}
\usepackage{amsfonts}
\usepackage{amsmath}
\usepackage{dsfont}
\usepackage{algorithm}
\usepackage{algorithmic}
\usepackage{breqn}
\usepackage{tikz}

% usado pra pseudo-codigo
\renewcommand{\algorithmicrequire}{\textbf{Input:}}
\renewcommand{\algorithmicensure}{\textbf{Output:}}

% operacoes do SVM
\DeclareMathOperator*{\argmax}{argmax}  
\DeclareMathOperator*{\argmin}{argmin}  

% coisinhas de desenho
\definecolor{light-gray}{gray}{0.95}
\usetikzlibrary{shapes.misc}
\tikzset{cross/.style={cross out, draw=black, minimum size=2*(#1-\pgflinewidth), inner sep=0pt, outer sep=0pt},
%default radius will be 1pt. 
cross/.default={5pt}}

\hypersetup{
	colorlinks=true,
	linkcolor=blue,
	urlcolor=red,
	linktoc=all
}
\begin{document}

\author{Lucas Romão Silva\\[1ex]
	\small Prof Dr. Roberto Hirata Jr.
}
\title{Análise de Sentimentos Aplicada à Política}



\frontmatter
\maketitle
\tableofcontents

\mainmatter
\chapter{Introdução}

A expansão da internet impulsionou o uso cada vez maior das redes sociais por cada vez mais
usuários. Segundo estudo realizado pelo portal Statisa\cite{statisa}, em 2017 existe
2,46 bilhões de usuários nas redes sociais do mundo todo.

As redes sociais tornaram-se um local para a manifestação de opinião dos usuários acerca de
diversos assuntos. Dentre as principais redes sociais, o twitter possui na comunidade brasileira
a maior quantidade de usuários fora dos Estados Unidos com 27,7 milhões de usuários.
Estudos recentes do Twitter Brasil mostram que o país foi o terceiro em maior crescimento em número
de usuários no ano de 2016, avançando em 18\% o número de usuários que utilizam a rede social ao
menos uma vez por mês em relação ao mesmo estudo realizado em 2015.\cite{twitterFolha}

Motivado pela popularidade do Twitter em uma parcela da população brasileira, aliada à facilidade
de obter-se dados postados na rede social decidiu-se analisar as opiniões dos usuários do Twitter
acerca do cenário político brasileiro atual construindo classificadores de tuítes onde cada
tuíte era classificado como opinião positiva, negativa ou neutra e estudar como
cada classificador e forma de representação dos dados contribui para a construção de um bom
classificador. 

Para isso, coletou-se tuítes de julho de 2016 a julho de 2017 sobre as reformas da previdência, lei
da terceirização, a proposta de emenda constitucional do teto dos gastos e sobre políticos
com notoriedade no cenário brasileiro atual como os ex-presidentes Luis Inácio Lula da Silva,
Dilma Rouseff e o senador Aécio Neves.

Além disto este trabalho se propôs a construir uma ferramenta (na forma de um site) que permite
facilitar a tarefa da classificação manual de dados textuais e a colaboração de mais usuários.

Este trabalho é organizado da seguinte forma: o capítulo 2 tratará da revisão dos estudos mais
recentes na literatura quanto ao uso do Twitter em tarefas de análise de sentimentos voltada à
política. O capítulo 3 trata da fundamentação teórica não só acerca dos métodos de aprendizado de
máquina, mas também técnicas de representação dos tuítes e um detalhamento do funcionamento e motivação
da criação da ferramenta para a classificação manual dos dados. No capítulo 4 são discutidos os resultados
dos algoritmos desenvolvidos e é realizada uma comparação destes com as implementações já existentes
na biblioteca \texttt{scikit} do Python. Por último no capítulo 5 são discutidos melhorias dos resultados
atuais e perspectivas futuras acerca deste trabalho.
\chapter{Revisão Bibliográfica}

Diversos estudos recentes têm sido realizados na área de análise de sentimentos.
Medhat (2014)\citep{medhat2014}, sumariza diversos estudos feitos dividindo-os em
abordagens (usando técnicas de aprendizado de máquina, dicionário léxico ou
híbrida) e em objetivos (classificação de sentimentos, detecção de emoções etc).

Em Pak (2010)\citep{pak2010} é analisado o uso de um corpus compostos de tuítes para
a realização de tarefas de análise de sentimentos e mineração de opinião. Nesse estudo
é construído um classificador de sentimentos para as opiniões associadas aos tuítes coletados
e é discutido desafios encontrados na hora de desenvolver um classificador para o corpus.

Outros estudos, assim como este trabalho, têm como foco o uso de análise de sentimentos
aplicados à política utilizando tuítes como o corpus. Em Tumasjan et. al (2010)\citep{tumasjan2010} é estudado
se é possível utilizar o Twitter para obter uma previsão para os resultados de uma eleição
tendo como base as eleições do parlamento alemão realizadas em 2009.

Risquandl e Petković (2013)\citep{petkovic2013} realizam experimentos tendo como cenário as
eleições presidenciais estadounidenses de 2012 para classificar sentimentos dos usuários do
twitter, porém diferente dos outros trabalhos anteriormente mencionados, esse estudo utiliza
técnicas de \textit{part-of-speech tagging} para identificar sobre quem os tuítes se referem e, a
partir disso, classifica o sentimento baseado a aspectos dessa entidade, no caso pautas eleitorais
que foram fortemente discutidas como casamento de pessoas do mesmo gênero, teaparty etc.

Bakliwal et. all (2013)\citep{bakliwal2013} realiza um estudo semelhante ao deste trabalho a classificação de opinião
tendo como base três classes (positivo, negativo ou neutro) a cerca de tuítes coletados sobre
as eleições gerais da Irlanda em fevereiro de 2011. A diferença entre os trabalhos é que Bakliwal
divide as classificações das opiniões entre cada partido.




\backmatter
% bibliography, glossary and index would go here.

\bibliographystyle{plain}
\bibliography{references.bib}{}

\end{document}