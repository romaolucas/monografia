\chapter{Introdução}

A expansão da internet impulsionou o uso cada vez maior das redes sociais por cada vez mais
usuários. Segundo estudo realizado pelo portal Statisa\cite{statisa}, existem
2,46 bilhões de usuários nas redes sociais do mundo todo no ano de 2017, o que contabiliza
quase metade da população mundial.

As redes sociais tornaram-se um local para a manifestação de opinião dos usuários acerca de
diversos assuntos. Dentre as principais redes sociais, o Twitter possui na comunidade brasileira
a maior quantidade de usuários fora dos Estados Unidos com 27,7 milhões de usuários.
Estudos recentes do Twitter Brasil mostram que o país foi o terceiro em maior crescimento em número
de usuários no ano de 2016, avançando em 18\% o número de usuários que utilizam a rede social ao
menos uma vez por mês em relação ao mesmo estudo realizado em 2015.\cite{twitterFolha}

O uso do Twitter como plataforma de expressão de opiniões acerca de diversos temas fez com que
muitas empresas utilizassem os dados da rede social para análise de informações relevantes. O IBM
Watson possui uma integração com o Twitter onde a ferramenta é capaz de coletar e analisar os dados
e revelar insights sobre os mesmos como localização, gênero dos usuários, polaridade dos tuítes
e palavras indicativas de cada polaridade. \cite{twitterWatson}

No mercado financeiro a rede social também é utilizada para captar informações relevantes como exemplo
da Bloomberg que disponibiliza uma ferramenta que monitora tuítes de empresas, executivos, blogs
financeiros e economistas e é possível gerar alertas para quando determinado tópico ou empresa recebe
muita atenção, uma vez que isso pode ser um sinal de movimentação no mercado de ações.	\cite{twitterBloomberg}

Motivado pela popularidade do Twitter em uma parcela da população brasileira, aliada à facilidade
de obter-se dados postados na rede social decidiu-se analisar as opiniões dos usuários do Twitter
acerca do cenário político brasileiro atual construindo classificadores de tuítes onde cada
tuíte era classificado como opinião positiva, negativa ou neutra e estudar como
cada classificador e forma de representação dos dados contribui para a construção de um bom
classificador.

Para conhecer o estado da arte na análise de sentimentos usando tuítes, fez-se uma revisão
bibliográfica dos artigos mais citados e mais relacionados ao trabalho, em especial aqueles
que tratavam do problema da análise de tuítes de política. Com os artigos pesquisados, teve-se uma
noção de possíveis problemas que poderiam ser encontrados no trabalho como a presença de ironia
que atrapalha a classificação e partes do texto que poderiam trazer mais informações relevantes
como as \textit{hashtags} e emojis. Ademais os artigos forneceram uma base de como pré-processar
o texto para então treinar os modelos.

Para isso, coletou-se tuítes de julho de 2016 a julho de 2017 sobre as reformas da previdência, lei
da terceirização, a proposta de emenda constitucional do teto dos gastos e sobre políticos
com notoriedade no cenário brasileiro atual como os ex-presidentes Luis Inácio Lula da Silva,
Dilma Rousseff e o senador Aécio Neves.

Ao longo deste trabalho foi desenvolvido uma ferramenta (CLAM) para auxiliar na tarefa da classificação 
manual dos dados e que estivesse disponível para ser utilizada por outras pessoas da comunidade
se assim quisessem. Disponibilizou-se também neste trabalho um tutorial de como subir uma instância
do CLAM no Heroku \ref{chap:clam}.

Além do CLAM, implementou-se dois classificadores, um utilizando o método de regressão logística
e outro utilizando Support Vector Machines (SVM) tanto para a versão binária da classificação quanto
para três classes (opinião positiva, negativa e neutra) e comparou o desempenho destas implementações
com as já disponibilizadas pelo \texttt{scikit} do Python a fim de verificar se ambas apresentavam mesma
acurácia e precisão sobre os dados de teste.

Após esta Introdução, no capítulo 2 é apresentada uma revisão bibliográfica dos artigos lidos que foram
mais relevantes para o trabalho, no capítulo 3 é definido os modelos de classificação que serão usados
bem como explica-se análise de sentimentos. No capítulo 4 são comentados os resultados dos experimentos
realizados e analisa-se o conjunto de dados. Por último no capítulo 5 são discutidos os resultados e
perspectivas futuras para o trabalho.