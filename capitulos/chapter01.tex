\chapter{Introdução}

A expansão da internet impulsionou o uso cada vez maior das redes sociais por cada vez mais
usuários. Segundo estudo realizado pelo portal statisa\cite{statisa}, em 2017 existe
2,46 bilhões de usuários nas redes sociais do mundo todo.

As redes sociais tornaram-se um local para a manifestação de opinião dos usuários acerca de
diversos assuntos. Dentre as principais redes sociais, o twitter possui na comunidade brasileira
a maior quantidade de usuários fora dos Estados Unidos com 27,7 milhões de usuários.
Estudo recentes do Twitter Brasil mostram que o país foi o terceiro em maior crescimento em número
de usuários no ano de 2016, avançando em 18\% o número de usuários que utilizam a rede social ao
menos uma vez por mês em relação ao mesmo estudo realizado em 2015.\cite{twitterFolha}

Motivado pela popularidade do Twitter em uma parcela da população brasileira, aliada à facilidade
de obter-se dados postados na rede social decidiu analisar as opiniões dos usuários do Twitter
acerca do cenário político brasileiro atual construindo um classificadores de tuítes onde cada
tuíte era classificado como contendo opinião positiva, negativa ou neutra e estudar como
cada classificador e forma de representar os dados contribui para a construção de um bom
classificador. 

Para isso, coletou-se tuítes de julho de 2016 a julho de 2017 sobre as reformas da previdência, lei
da terceirização, a proposta de emenda constitucional do teto dos gastos e sobre políticos
com notoriedade no cenário brasileiro atual como os ex-presidentes Luis Inácio Lula da Silva,
Dilma Rouseff e o senador Aécio Neves.

Além disto este trabalho se propôs a construir uma ferramenta (na forma de um site) que permite
facilitar a tarefa da classificação manual de dados textuais e a colaboração de mais usuários.

Este trabalho é organizado da seguinte forma: o capítulo 2 tratará da revisão dos estudos mais
recentes na literatura quanto ao uso do Twitter em tarefas de análise de sentimentos voltada à
política. O capítulo 3 trata da fundamentação teórica não só acerca dos métodos de aprendizado de
máquina, mas também técnicas de representação dos tuítes e um detalhamento do funcionamento e motivação
da criação da ferramenta para a classificação manual dos dados. No capítulo 4 é discutido os resultados
dos algoritmos desenvolvidos e é realizada uma comparação destes com as implementações já existentes
na biblioteca \texttt{scikit} do Python. Por último no capítulo 5 é discutido melhorias dos resultados
atuais e perspectivas futuras acerca deste trabalho.