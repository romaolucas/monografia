\chapter{Considerações finais}

Neste trabalho, estudou-se a classificação de tuítes de política e, para isso,
estudou-se a literatura a respeito do problema análise de sentimentos e modelos de
aprendizado de máquina que poderiam ser usados para realizar a tarefa.

Ao fim deste trabalho, implementou-se versões do SVM e da regressão logística
tanto para o caso binário quanto o de várias classes e, ao fim dos experimentos,
teve-se que os modelos apresentaram uma acurácia de $68,3\%$ para o caso
de três classes e $70,4\%$ no caso binário, valores próximos aos
obtidos pelas implementações da biblioteca \texttt{scikit-learn}, além de apresentar
valores médios próximos para as outras médias avaliadas.

Além da implementação dos classificadores e dos experimentos realizados, desenvolveu-se o CLAM,
ferramenta \textit{open source} criada com o intuito de facilitar a etapa de classificação manual
das opiniões de textos, disponível em \url{https://github.com/romaolucas/manual-classifier-helper}.
Em \ref{chap:clam} foi escrito um tutorial de como subir uma instância do CLAM no Heroku.

Ao fim das rodadas de experimentos, obteve-se um bom classificador
para as classes mais presentes do nosso conjunto de dados que conciliava
não só uma boa acurácia, mas também com um bom compromisso entre precisão e
revocação. Futuras melhorias a serem feitas neste classificador seria utilizar
seleção de características antes de treinar o modelo para assim mantermos apenas
as \textit{features} mais relevantes para cada classe e não só melhorar a acurácia
do estimador, mas também o tempo de convergência de cada modelo.

Quanto ao problema com três classes é possível tentar melhorar a revocação e pontuação
f1 para a classe positiva introduzindo mais dados dessa classe ao nosso conjunto de dados
e dar um peso maior a elementos que pertençam a esta classe para assim garantir uma diminuição
da taxa de falso negativos, entretanto é importante ressaltar que consequentemente diminuiria
a precisão do algoritmo pois com mais elementos classificados como positivos, maior a chance
de falso positivos.

Para o CLAM uma possível melhoria a ser implementada seria a possibilidade de mais usuários avaliarem
um mesmo conjunto de textos para garantir um consenso na hora de associar uma opinião a um texto.
Importante ressaltar que o sistema já foi desenvolvido preparado para suportar essa funcionalidade,
bastaria modificar a listagem de textos a serem classificados para incluir aqueles que já possuem
uma avaliação e aqueles que não possuem ainda.

Percebeu-se ao longo deste trabalho que mais importante do que um bom entendimento
do funcionamento de cada classificador é analisar melhor os dados e achar elementos
mais relevantes para cada classe para assim conseguir melhorar a representação dos dados
como um vetor de \textit{features} e, consequentemente, garantir melhores resultados.

Outra melhoria importante a ser realizada é ter mais tuítes classificados, para assim conseguir
obter resultados mais expressivos e ter um conjunto de testes maior.

Do ponto de onde o trabalho é finalizado há espaço para não só melhorias quanto as discutidas
nos parágrafos anteriores, mas também é possível aplicar técnicas de \textit{part-of-speech tagging}
\cite{pos2017}
para descobrir não só a polaridade da opinião, mas também o que falaram acerca de cada entidade
facilitando o uso da ferramenta para obter melhor entendimento acerca das discussões dos usuários.
Além disso, classificar mais dados manualmente para garantir maior eficácia nos classificadores
junto com permitir que alguns tuítes já classificados tenham uma segunda opinião a fim de
diminuir a ambiguidade presente em alguns documentos também são melhorias a serem feitas.
