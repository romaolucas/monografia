\chapter{Considerações finais}

Ao fim das rodadas de experimentos, obteve-se um bom classificador
para as classes mais presentes do nosso conjunto de dados que conciliava
não só uma boa acurácia, mas também com um bom compromisso entre precisão e
revocação. Futuras melhorias a serem feitas neste classificador seria utilizar
seleção de características antes de treinar o modelo para assim mantermos apenas
as \textit{features} mais relevantes para cada classe e não só melhorar a acurácia
do estimador, mas também o tempo de convergência de cada modelo.

Quanto ao problema com três classes é possível tentar melhorar a revocação e pontuação
f1 para a classe positiva introduzindo mais dados dessa classe ao nosso conjunto de dados
e dar um peso maior a elementos que pertençam a esta classe para assim garantir uma diminuição
da taxa de falso negativos, entretanto é importante ressaltar que consequentemente diminuiria
a precisão do algoritmo pois com mais elementos classificados como positivos, maior a chance
de falso positivos.

Percebeu-se ao longo deste trabalho que mais importante do que um bom entendimento
do funcionamento de cada classificador é analisar melhor os dados e achar elementos
mais relevantes para cada classe para assim conseguir melhorar a representação dos dados
como um vetor de \textit{features} e, consequentemente, garantir melhores resultados.
