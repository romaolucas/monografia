\chapter{Materiais e Métodos}

\section{Considerações}
\label{sec:considerations}

Ao longo deste capítulo, se usará n para se referir à quantidade de elementos
fornecidas ao nosso modelo, cada entrada é i é um vetor $x_i \in \mathbb{R}^m$.
A entrada será referida como X a efeitos de conta e assim cada entrada será 
dada como $X_i$ nesse caso.
Para cada i associaremos duas variáveis $t_i$ e $y_i$ que se referem ao valor
esperado e ao valor obtido através do treinamento, respectivamente.

A notação $\mathds{1}_{i == j}$ é uma função indicadora que vale 1 se i é
igual a j e 0 caso contrário.

\section{Contextualização}
\label{sec:methods}

Os problemas tratados por \textit{Machine Learning} classificam-se de forma
geral em três tipos:

\begin{itemize}
	\item Aprendizado supervisionado: nesse caso tem-se os elementos de entrada e
	para cada um desses elementos, tem-se associado um rótulo $t_i$. Nesse caso o modelo
	deve ser treinado com base nos elementos dados para que se possa prever o rótulo %não consegui pensar numa tradução boa pra label%
	de uma nova entrada;
	\item Aprendizado não-supervisionado: nesse caso tem-se apenas os elementos de entrada. 
	O objetivo deste tipo de problema é tentar modelar uma distribuição ou estrutura comum
	entre os dados para que se possa entendê-los melhor;
	\item Aprendizado semi-supervisionado: nesse último caso alguns elementos possuem um rótulo
	associado. Problemas desse tipo aplicam técnicas gtanto de aprendizado supervisionado como
	de não-supervisionado.
\end{itemize}

Neste trabalho será tratado um problema de aprendizado supervisionado que é o da classificação.

Na classificação temos k classes e cada elemento i da entrada é associado a uma classe $t_i = \{1..k\}$.
O objetivo do problema da classificação é dado entrada $X = (x_1, x_2, \ldots, x_n)$ 
e $t = (t_1, \ldots, t_n)$ treinar um modelo capaz de prever classes para um x qualquer.

Há diversos algoritmos na literatura que se propõem a resolver o problema da classificação.
Bishop (2006)\cite{bishop2006} enuncia diversos dos algoritmos comumente utilizados para a
classificação, cada algoritmo possui seus prós e contras e utiliza diferentes abordagens.

Para este trabalho escolheu-se implementar os algoritmos \textit{Logistic Regression} e
\textit{Support Vector Machines}, que será chamado simplesmente de SVM por facilidade.

Tanto para o \textit{Logistic Regression} quanto SVM será explicado a princípio o problema
será inicialmente abordado a partir da classificação binária e, a partir dela, será descrito
como estender para o problema com mais de duas classes, que será é o caso deste trabalho.

\section{Logistic Regression}
\label{sec:logreg} 

O nosso modelo será construído de forma probabilística, isto é, a partir de um 
discriminante linear $w^Tx + w_0$ atribuiremos uma probabilidade de um elemento x
pertencer à classe $C^1$ e, consequentemente a probabilidade de pertencer à classe
$C^2$ é dada por $1 - P(C^1 | x)$. O termo $w_0$ é chamado de viés, e para efeito
das contas que serão feitas consideraremos vetores w' e x' da forma $x' = (x, 1)$,
$w'= (w, w_0)$ note entretanto que os chamaremos daqui pra frente simplesmente de w e x.

No caso da classificação binária, usaremos que $t_n \in \{0, 1\}$ onde $t_n = 1$ se o
elemento pertence à classe $C^1$ e $t_n = 0$ se pertence à classe $C^2$.

A classificação de um elemento será a classe a qual ele tem maior probabilidade de pertencer.

Para utilizarmos nosso discriminante para atribuir as probabilidades, utiliza-se a função
sigmóide definida por:

\begin{center}
	\begin{equation}
		\sigma(a) = \frac{1}{1 + exp(-a)}
	\end{equation}
\end{center}

A função sigmóide é usada para obter a probilidade por ter a propriedade de mapear
todo o conjunto dos números reais dentro do intervalo $[0, 1]$.

Com $exp$ sendo a função exponencial. Aplicando ao nosso modelo obtêm-se a expressão:

\begin{center}
	\begin{equation}
		P(C^1 | x) = y(x) = \sigma(w^Tx)
	\end{equation}
\end{center}

Importante notar que apesar de utilizarmos o vetor x nas equações, é possível aplicarmos uma
transformação linear $\phi : \mathcal{R}^m \rightarrow \mathcal{R}^d$ à entrada x para obtermos
$\phi(x)$. O uso de transformação linear no nosso conjunto de entrada nos permite transformar o 
domínio para que se obtenha uma separação melhor entre as classes ou até mesmo fazer a redução
da dimensão do domínio.

Com essa equação em mãos, nosso objetivo é minimizar o erro na classificação dos dados. Tomamos
como erro o negativo do logaritmo da verosimilhança de nossa função que é dada por:

\begin{center}
	\begin{equation}
		E(w) = - \sum_{i = 1}^{n} p(t | w) = 
		- \sum_{i = 1}^{n} \{ t_nln(y_n) + (1 - t_n) ln(1 - y_n) \}
	\end{equation}
\end{center}

A fim de minimizar o erro, utiliza-se métodos de otimização linear (note que por mais que se use uma
transformação linear $\phi$ sobre x nosso problema ainda é linear sobre w).

Dois métodos são comumente	usados: método do gradiente e método de Newton-Raphson.
Esses métodos são utilizados tanto para o caso da classificação binária
quanto o caso da classificação com $k > 2$. A diferença entre um problema e outro será abordada
com mais especificidade a seguir.

Uma dúvida natural que surge ao ter que resolver um problema de otimização é o caso de parar o
procedimento em um mínimo local ao invés de um mínimo local da função.	Entretanto, temos que nossa
função $E(w)$ é côncava, isto é, $E(\lambda w + (1 - \lambda ) w') = \lambda E(w) 
	+ (1 - \lambda ) w'$
 $\forall w, w' \in R^m, \lambda \in [0, 1]$, tal propriedade nos garante que existe um único minizador.
 

\subsection{Método do Gradiente}\label{subsec:grad_descent}

Para este método, minimiza-se a função objetivo, no caso $E(w)$ utilizando apenas o gradiente
da função junto de um passo $\alpha$. Com ambos valores em mãos, o valor w é atualizado usando
a equação:

\begin{center}
	\begin{equation}
		w^{ ( novo )} = w^{ (antigo) }  + \alpha \nabla E(w)
	\end{equation}
\end{center}

Com $\nabla E(w)$ sendo o gradiente do vetor de pesos. O gradiente é calculado usando o fato de que
a derivada da função sigmóide com respeito a um vetor a é dada por:

\begin{center}
	\begin{equation}
	\label{eq:sigmoid_derivative}
		\frac{d \sigma}{d a} = \sigma (1 - \sigma )
	\end{equation}
\end{center}

Usando \ref{eq:sigmoid_derivative} tem-se a seguinte equação para o gradiente:

\begin{center}
	\begin{equation}\label{eq:gradient}
		\nabla E(w) = X^T(y - t)
	\end{equation}
\end{center}

Onde $y_n = P(C^1 | x_n) = \sigma(w^Tx)$ e $t_n$ tal qual assumido no começo da seção.

O algoritmo de atualização do vetor de pesos descrito a seguir vale tanto para o método
do gradiente quanto para o de Newton-Raphson, portanto para o segundo será focado apenas nas
diferenças entre os dois.


\begin{algorithm}[H]
	\caption{Logistic Regression usando método do gradiente}
	\begin{algorithmic}[1]
		\REQUIRE Matriz $ X \in \mathbb{R}^{n \times m} $, 
		vetor de rótulos $t \in \{0, 1\}^n$
		\ENSURE Vetor de pesos $w \in \mathbb{R}^m$
		\STATE $iteracao \leftarrow 0$
		\STATE $w \leftarrow 0$
		\WHILE{ $|E(w)^{ (iteracao) } - E(w)^{ (iteracao - 1) } | \ge \epsilon$ \AND
		$iteracao < maxIteracoes$ } \label{lst:line:condition}
			\STATE $y \leftarrow (\sigma(w^Tx_1), \sigma(w^Tx_2), \ldots, \sigma(w^Tx_n))^T$
			\STATE $\nabla E(w) \leftarrow X^T(y - t)$
			\STATE $w \leftarrow w - \alpha \nabla E(w)$
			\STATE $E(w)^{ (iteracao) } \leftarrow 
			- \sum_{i = 1}^{n} \{ t_nln(y_n) + (1 - t_n) ln(1 - y_n) \}$
			\STATE $iteracao \leftarrow iteracao + 1$
		\ENDWHILE
	\end{algorithmic}
\end{algorithm}

Importante notar que em~\ref{lst:line:condition} tem-se duas condições de paradas do algoritmo que
são o número de iterações e a diferença da diminuição da função objetivo for menor do que
um dado $\epsilon$. Tais condições são chamadas de condições de convergência e nos garantem
que chegamos a um valor suficientemente próximo do ótimo, uma vez que atingir este valor
pode exigir um número muito alto de iterações, o que traz um custo computacional.
 Na implementação do algoritmo, escolheu-se um valores
padrão para $\epsilon$ e $maxIteracoes$ como $10^{-4}$ e $200$ respectivamente.

A quantidade de iterações necessárias para a convergência é influenciada fortemente pela
escolha de $\alpha$. Um valor pequeno para $\alpha$ acarretaria em muitas iterações até
a convergência ao passo que um valor muito grande pode fazer com que se pare muito longe
do valor ótimo.


\subsection{Método de Newton-Raphson}
\label{subsec:newton-raphson}

Vimos em \ref{subsec:grad_descent} que o método do gradiente apesar de implementação
simples pode levar muito tempo para resolver o problema.

O método de Newton-Raphson acaba convergindo mais rápido do que o método do gradiente,
contudo ao custo de uma maior complexidade devido à necessidade de calcular outros
elementos.

A atualização agora é feita seguindo a equação

\begin{center}
	\begin{equation}\label{eq:newton-raphson}
		w^{ (novo) } = w^{ (antigo) } - H^{-1} \nabla E(w)
	\end{equation}
\end{center}

Onde H é a matriz Hessiano da função erro, que é calculado usando $H = \nabla \nabla E(w)
= X^TRX$ onde R é uma matriz diagonal $n \times n$ onde as entradas da diagonal principal
valem $R_{kk} = y_k(1 - y_k)$. Substituindo os valores de H e usando
\ref{eq:gradient} em \ref{eq:newton-raphson} obtemos


\begin{equation}
\begin{split}
w^{ (novo) } & = w^{ (antigo) } - (X^T R X)^{-1} \nabla E(w) \\
	& = (X^T R X)^{-1}[(X^T R X)w^{ (antigo) } - X^T(y - t)]  
\end{split}
\end{equation}

\subsection{Extensão para o caso de várias classes}

Diversas abordagens podem ser usadas para resolver o problema multiclasse, no
caso será usado diversos discriminantes $y_k$ com $k = \{1, \ldots K\}$ com K
sendo o total de classes. Assim nosso vetor w agora é uma matriz
$W \in \mathbb{R}^{m \times k}$. 

Quanto à codificação do vetor de rótulos, 
segue-se a codificação dada em Bishop (2006)\cite{bishop2006} de $1-K$,
na codificação tem-se que $t_n \in \{0, 1\}^k$ com $t_{nk} = 1$ se o elemento
n pertencer à classe k e 0 nas demais entradas. 

Quanto a função de probabilidade que desejamos estimar, utiliza-se a função
\textit{softmax} que é dada pela equação:

\begin{center}
	\begin{equation}
		P(C^k | x_n) = y_{nk} = \frac{exp(w_k^Tx_n)}{\sum_j exp(w_j^Tx_n)} 
	\end{equation}
\end{center}

Que nos dá verossimilhança e consequentemente a seguinte função de erro, tomada
usando o negativo do logaritmo da verossimilhança.

\begin{center}
	\begin{align*}
				P(T | w_1, \ldots, w_k) &= \prod_{i = 1}^{n} \prod_{j = 1}^{k} P(C^j | x_i)^{t_{ij}} = \prod_{i = 1}^{n} \prod_{j = 1}^{k} y_{ij}^{t_{ij}} \\
		E(W) &= - \sum_{i = 1}^{n} \sum_{j = 1}^{k} t_{ij} ln(y_{ij})	
	\end{align*}
\end{center}

Novamente nesse caso pode-se encontrar o valor de W que minimize $E(W)$ usando os
dois métodos discutidos em \ref{subsec:grad_descent} e \ref{subsec:newton-raphson},
porém agora temos que a derivada com respeito a cada $w_k^Tx$ vale:

\begin{center}
	\begin{equation}\label{eq:softmax_derivative}
		\frac{\partial y_k}{\partial (w_j^Tx)} = y_k(\mathds{1}_{k == j} - y_j)
	\end{equation}
\end{center}

Nosso valor de W pode ser interpretado tanto como uma matriz $m \times k$ como um
único vetor $1 \times mk$ onde $W = (w_1, w_2, \ldots, w_k)$.

Usando essa representação, podemos calcular o vetor gradiente onde a derivada
com respeito a cada $w_j$ é dada pela equação:

\begin{center}
	\begin{equation}
		\nabla_{w_j} E(W) = X^T(Y_j - T_j)
	\end{equation}
\end{center}

Com $Y_j$ e $T_j$ correspondendo, respectivamente, às j-ésimas colunas de Y e T.

Com o gradiente em mãos já temos o que é necessário para o método do gradiente e a
atualização seria feita da forma $W^{ (novo) } = W^{ (antigo) } - \alpha \nabla E(W)$.

Para aplicarmos o método de Newton-Raphson, seria necessário computarmos o Hessiano que
nesse caso seria uma matriz $m*k \times m*k$ com cada bloco $j, i$ contendo uma matriz
$m \times m$ calculada pela equação:

\begin{center}
	\begin{equation}
		\nabla_{w_i} \nabla_{w_j} E(W) = - \sum_{k = 1}^n y_{ki}( \mathds{1}_{i == j} - y_{kj})
		X_k^TX_k
	\end{equation}
\end{center} 

Onde $X_k$ é a k-ésima linha de X. Com essas equações em mãos nossa atualização de
W seria feita usando a fórmula $W^{ (novo) } = W^{ (antigo) } - H^{-1}\nabla E(W)$.

A classificação de um novo x é feita a partir
do cálculo de $P(C^k | x) = y_k(x), \forall k = \{1, \ldots, k\}$.

A classe de x é dada pelo k que tiver a maior probabilidade sobre os demais.


\section{Support Vector Machine}

Assim como fizemos com o logistic regression, começaremos com a definição
para o caso binário e depois iremos estender para mais de uma classe. Nesse caso
nossas classes serão $t_n \in \{-1, 1\}$ onde $t_n = 1$ se x pertence à classe $C^1$ e
$t_n = -1$ se x pertence à classe $C^2$.

No algoritmo SVM a classificação é feita a partir de um discriminante linear
da forma 

\begin{center}
	\begin{equation}\label{eq:svm-discriminant}
		y(x) = w^Tx + b
	\end{equation}
\end{center}

Tal y é chamado de hiperplano de decisão e a classificação é baseada no sinal de
y. Se $y(x) > 0$, x é atribuído à classe $C^1$, caso contrário é atribuído à classe
$C^2$.

Porém ao invés de procurarmos um w que separe perfeitamente todas as classes (que
não necessariamente existe), nosso objetivo é maximizar a margem do discriminante
linear, isto é, a menor distância de um ponto ao hiperplano. A distância de um ponto ao
hiperplano é dado pela fórmula

\begin{center}
	\begin{equation}
		\frac{[t_n(w^Tx_n + b)]}{||w||}
	\end{equation}
\end{center}
 
Na descrição inicial do problema vamos tratar o caso com o conjunto X linearmente separável, 
o que indica que é possível obter um hiperplano que separe sem erro todas as classes para, em
seguida, tratarmos o caso real que é o do conjunto que não é linearmente separável.
O fato de o conjunto ser linearmente separável nos garante que $t_ny(x_n) >0 \forall n$.


\begin{center}
	\begin{equation}
		\argmax_{x, b} \left\{ \frac{1}{||w||} min_n [t_n(w^Tx_n + b)] \right\}
	\end{equation}
\end{center}


Podemos ajustar w e b de forma a termos que $t_n(w^Tx_n + b) = 1$ para o ponto mais
próximo da margem e $t_n(w^Tx_n + b) \ge 1, \forall n$. Com esse reajuste temos que nosso
problema de encontrar um vetor de pesos de margem maximizada seria de maximizar 
$\frac{1}{||w||}$ que é equivalente ao problema de otimização quadrática:

\begin{center}
	\begin{equation}
		\begin{array}{ll@{}ll}
				\argmin_{w, b} & ||w||^2 &\\
				\textbf{sujeito a}& t_k(w^Tx_k + b), &&k = 1, \ldots ,n 
		\end{array}
	\end{equation}
\end{center}

Agora iremos supor que não necessariamente nossa entrada não é linearmente separável, 
isto é, não existe um hiperplano que separe perfeitamente as duas classes. Assim
iremos permitir que alguns valores estejam classificados incorretamente, para isso
será necessário suavizarmos nossa margem penalizando cada uma das entradas incorretamente
classificadas. Cortes e Vapnik (1995)\cite{cortesVapnik1995} descrevem a penalização
através da introdução de variváveis de folga $\xi_n \ge 0$ para cada elemento de X.
Um elemento corretamente classificado teŕa $\xi_n = 0$, os demais pontos têm
 $\xi_n = |t_n - y(x_n)|$.
 
Com isso nossa restrição de $t_ny(x_n) \ge 1$ é modificada e tem-se 
$t_ny(x_n) \ge 1 - \xi_n \forall n$.

O valor de $\xi_n$ assim nos indica três possíveis casos:

\begin{itemize}
	\item Se $\xi_n = 0$, $x_n$ está corretamente classificado e se encontra
	ou na margem ou do lado correto dela.
	\item Se $0 < \xi_n \le 1$, $x_n$ está corretamente classificado e se encontra
	entre a margem e o hiperplano.
	\item Se $\xi_n > 1$, $x_n$ não está classificado corretamente.
\end{itemize}

A função objetivo agora precisa conter os valores de $\xi$ e para isso colocamos
uma constante $C > 0$ que define a compensação entre a penalização das variáveis de
folga e a margem. Com isso temos o seguinte problema de otimização quadrática:

\begin{center}
	\begin{equation}
		\begin{array}{ll@{}ll}
			\argmin_{w, b, \xi} & C\sum_{i = 1}^n\xi_i + \frac{1}{2}||w||^2 &\\
			\text{sujeito a}& t_k(w^Tx_k + b) \ge 1 - \xi_k, && k = 1, \ldots, n
		\end{array}
	\end{equation}
\end{center}

Para implementar o SVM basta então resolver o problema de otimização quadrática
acima. Isto é feito seguindo os seguintes passos

\begin{enumerate}
	\item Introduz-se multiplicadores de Lagrange $\alpha_n$ e $\mu_n$ e obtem-se o 
	Lagrangiano com as restrições
		\begin{dgroup}
			\begin{dmath}
				L(w, b, \xi) = \frac{1}{2}||w||^2 + C\sum_{i = 1}^n\xi_i - \sum_{i = 1}^n \alpha_i\{t_i(w^Tx_i + b) - 1 + \xi_i\} - \sum_{i = 1}^n \mu_i \xi_i
			\end{dmath}
			\begin{dmath}
				\alpha_n \ge 0
			\end{dmath}
			\begin{dmath}
				t_ny(x_n) - 1 + \xi_n \ge 0
			\end{dmath}
			\begin{dmath}
				\alpha_n\{t_n(w^Tx_n + b) - 1 + \xi_n\} = 0
			\end{dmath}
			\begin{dmath}
				\mu_n \ge 0
			\end{dmath}
			\begin{dmath}
				\mu_n \xi_n = 0
			\end{dmath}
		\end{dgroup}
	\item Derivamos o lagrangiano com respeito a w, b e $\xi$ e igualamos a 0 para obtermos 
	os valores ótios para essas variáveis e, assim obtemos os seguintes valores:
		\begin{gather}
				\frac{\partial L}{\partial w} = 0 \Rightarrow w = \sum_{i = 1}^n \alpha_i t_i x_i \\
				\frac{\partial L}{\partial b} = 0 \Rightarrow \sum_{i = 1}^n \alpha_i t_i  = 0 \\
				\frac{\partial L}{\partial \xi} = 0 \Rightarrow \alpha_n = C - \mu_n 
		\end{gather} 
	\item Com isso em mãos, resolve-se não o problema primal e sim o dual (utiliza-se aqui o fato 
	de que se as condições mencionadas no item anterior são satisfeitas, tem-se que vale
	a dualidade forte e o valor ótimo de ambas as funções coincide). O dual é dado pelo problema
		\begin{center}
			\begin{equation}
				\begin{aligned}	
				& \underset{\alpha}{\text{min}}
				& & \tilde{L}(\alpha) = \sum_{i = 1}^n \alpha_i - 1/2 \sum_{i = 1}^n \sum_{j = 1}^n \alpha_i\alpha_j t_i t_j k(x_i, x_j) \\
				& \text{sujeito a}
				& & 0 \le \alpha_i \le C, i = 1, \ldots, n \\
				&&& \sum_{i = 1}^n \alpha_i t_i = 0 
				\end{aligned}
			\end{equation}
		\end{center}
	No problema a cima é importante destacar a expressão $k(x_i, x_j)$, essa expressão é um
	\textit{Kernel} que é uma função onde $k(x, x') = \phi(x)^T\phi(x')$ com $\phi$ sendo
	alguma transformação linear. Tal qual no caso da regressão logística, essas
	transformações são usadas para mudar o domínio da entrada x.
	\item Acha-se o valor de b usando a fórmula:
		\begin{center}
			\begin{equation}
				b = \frac{1}{N_{\mathcal{M}}} \sum_{n \in \mathcal{M}} \left( t_n - \sum_{m \in \mathcal{S}} \alpha_m t_m k(x_n, x_m) \right)
			\end{equation}
		\end{center}
	Com $\mathcal{M}$ sendo o conjunto de pontos que satisfazem $0 < \alpha_n < C$ e 
	$\mathcal{S}$ o conjunto de vetores de suporte (pontos que possuem $\alpha_n > 0$ e,
	consequentemente, contribuem para a classificação do modelo).
\end{enumerate}

Uma vez resolvido o problema dual e encontrado valor de b, podemos classificar um novo
x usando o sinal do discriminante $y(x)$ dado por
\begin{center}
	\begin{equation}
		y(x) = \sum_{i = 1}^n \alpha_i t_i k(x, x_i) + b = \sum_{n \in \mathcal{S}} \alpha_n t_n l(x, x_n) + b
	\end{equation}
\end{center}

\subsection{Extensão ao caso de multiclasses}

Diversas abordagens são possíveis para o caso de multiclasses. A escolhida entre elas
foi o método intuitivo chamado \textit{One-versus-all} (OVA). Nesse método é construído
K classificadores, com K sendo o número de classes. Cada classificador $y_k$ define uma
superfície de decisão que separa a classe k das demais (por isso o nome \textit{One-versus-all}).
Um novo x tem sua classe dada pela que o classificador $y_k$ tem maior valor, isto é

\begin{center}
	\begin{equation}
		y(x) = \argmax_{k} y_k(x)
	\end{equation}
\end{center}

Portanto a classificação multiclasses se utiliza de todos os recursos já apresentados no caso
binário o que torna simples a construção do classificador.

Na imagem abaixo é mostrado a ideia por trás do método OVA.

