\chapter{Materiais e Métodos}

\section{Considerações}
\label{sec:considerations}

Ao longo deste capítulo, se usará n para se referir à quantidade de elementos
fornecidas ao nosso modelo, cada entrada é i é um vetor $x_i \in \mathbb{R}^m$.
A entrada será referida como X a efeitos de conta e assim cada entrada será 
dada como $X_i$ nesse caso.
Para cada i associaremos duas variáveis $t_i$ e $y_i$ que se referem ao valor
esperado e ao valor obtido através do treinamento, respectivamente.

A notação $\mathds{1}_{i == j}$ é uma função indicadora que vale 1 se i é
igual a j e 0 caso contrário.

\section{Contextualização}
\label{sec:methods}

Os problemas tratados por \textit{Machine Learning} classificam-se de forma
geral em três tipos:

\begin{itemize}
	\item Aprendizado supervisionado: nesse caso tem-se os elementos de entrada e
	para cada um desses elementos, tem-se associado um rótulo $t_i$. Nesse caso o modelo
	deve ser treinado com base nos elementos dados para que se possa prever o rótulo %não consegui pensar numa tradução boa pra label%
	de uma nova entrada;
	\item Aprendizado não-supervisionado: nesse caso tem-se apenas os elementos de entrada. 
	O objetivo deste tipo de problema é tentar modelar uma distribuição ou estrutura comum
	entre os dados para que se possa entendê-los melhor;
	\item Aprendizado semi-supervisionado: nesse último caso alguns elementos possuem um rótulo
	associado. Problemas desse tipo aplicam técnicas gtanto de aprendizado supervisionado como
	de não-supervisionado.
\end{itemize}

Neste trabalho será tratado um problema de aprendizado supervisionado que é o da classificação.

Na classificação temos k classes e cada elemento i da entrada é associado a uma classe $t_i = \{1..k\}$.
O objetivo do problema da classificação é dado entrada $X = (x_1, x_2, \ldots, x_n)$ 
e $t = (t_1, \ldots, t_n)$ treinar um modelo capaz de prever classes para um x qualquer.

Há diversos algoritmos na literatura que se propõem a resolver o problema da classificação.
Bishop (2006)\cite{bishop2006} enuncia diversos dos algoritmos comumente utilizados para a
classificação, cada algoritmo possui seus prós e contras e utiliza diferentes abordagens.

Para este trabalho escolheu-se implementar os algoritmos \textit{Logistic Regression} e
\textit{Supor Vector Machines}, que será chamado simplesmente de SVM por facilidade.

Tanto para o \textit{Logistic Regression} quanto SVM será explicado a princípio o problema
será inicialmente abordado a partir da classificação binária e, a partir dela, será descrito
como estender para o problema com mais de duas classes, que será é o caso deste trabalho.

\section{Logistic Regression}
\label{sec:logreg} 

Para classificar um dado elemento x entre as possíveis classes $C_1$ e $C_2$, é utilizado
um discriminante linear da forma $y(x) = w^Tx + w_0$ sendo w o vetor de pesos associado.
A classe atribuída a um vetor x é baseado no sinal de $y(x)$, se $y(x) \ge 0$ ele é designado
à classe $C^1$, caso contrário é designado à classe $C^2$.

No caso da classificação binária, usamos que $t_n \in \{0, 1\}$

Nesse caso, diz-se que uma superfície de decisão é definida pelo hiperplano $y(x) = 0$ onde
sua posição é determinada pelo elemento $w_0$ que chamaremos de viés. Uma vez que tanto nosso
vetor de pesos w quanto nossos vetores x do conjunto de treino possuem m elementos, iremos criar
vetores novos $w' = (w_0, w), x' = (1, x)$.

O nosso modelo será construído de forma probabilística, uma vez que o objetivo será obter um vetor
w de modo que possamos estimar as probabilidades condicionais $P(C^1 | x)$ e consequentemente
$P(C^2 | x) = 1 - P(C^1 | x)$, isto é, a probabilidade de um vetor x pertencer à uma determinada 
classe. 

Para utilizarmos nosso discriminante $y(x)$ para atribuir as probabilidades, utiliza-se a função
sigmóide definida por:

\begin{center}
	\begin{equation}
		\sigma(a) = \frac{1}{1 + exp(-a)}
	\end{equation}
\end{center}

Com $exp$ sendo a função exponencial. Aplicando ao nosso modelo obtêm-se a expressão:

\begin{center}
	\begin{equation}
		P(C^1 | x) = y(x) = \sigma(w^Tx)
	\end{equation}
\end{center}

Importante notar que apesar de utilizarmos o vetor x nas equações, é possível aplicarmos uma
transformação linear $\phi : \mathcal{R}^m \rightarrow \mathcal{R}^d$ à entrada x para obtermos
$\phi(x)$. O uso de transformação linear no nosso conjunto de entrada nos permite transformar o 
domínio para que se obtenha uma separação melhor entre as classes ou até mesmo fazer a redução
da dimensão do domínio.

Com essa equação em mãos, nosso objetivo é minimizar o erro na classificação dos dados. Tomamos
como erro o negativo do logaritmo da verosimilhança de nossa função que é dada por:

\begin{center}
	\begin{equation}
		E(w) = - \sum_{i = 1}^{n} p(t | w) = 
		- \sum_{i = 1}^{n} \{ t_nln(y_n) + (1 - t_n) ln(1 - y_n) \}
	\end{equation}
\end{center}

A fim de minimizar o erro, utiliza-se métodos de otimização linear (note que por mais que se use uma
transformação linear $\phi$ sobre x nosso problema ainda é linear sobre w).

Dois métodos são comumente	usados: método do gradiente e método de Newton-Raphson.
Esses métodos são utilizados tanto para o caso da classificação binária
quanto o caso da classificação com $k > 2$. A diferença entre um problema e outro será abordada
com mais especificidade a seguir.

Uma dúvida natural que surge ao ter que resolver um problema de otimização é o caso de parar o
procedimento em um mínimo local ao invés de um mínimo local da função.	Entretanto, temos que nossa
função $E(w)$ é côncava, isto é, $E(\lambda w + (1 - \lambda ) w') = \lambda E(w) 
	+ (1 - \lambda ) w'$
 $\forall w, w' \in R^m, \lambda \in [0, 1]$, tal propriedade nos garante que existe um único minizador.
 

\subsection{Método do Gradiente}\label{subsec:grad_descent}

Para este método, minimiza-se a função objetivo, no caso $E(w)$ utilizando apenas o gradiente
da função junto de um passo $\alpha$. Com ambos valores em mãos, o valor w é atualizado usando
a equação:

\begin{center}
	\begin{equation}
		w^{ ( novo )} = w^{ (antigo) }  + \alpha \nabla E(w)
	\end{equation}
\end{center}

Com $\nabla E(w)$ sendo o gradiente do vetor de pesos. O gradiente é calculado usando o fato de que
a derivada da função sigmóide com respeito a um vetor a é dada por:

\begin{center}
	\begin{equation}
	\label{eq:sigmoid_derivative}
		\frac{d \sigma}{d a} = \sigma (1 - \sigma )
	\end{equation}
\end{center}

Usando \ref{eq:sigmoid_derivative} tem-se a seguinte equação para o gradiente:

\begin{center}
	\begin{equation}\label{eq:gradient}
		\nabla E(w) = X^T(y - t)
	\end{equation}
\end{center}

Onde $y_n = P(C^1 | x_n) = \sigma(w^Tx)$ e $t_n$ tal qual assumido no começo da seção.

O algoritmo de atualização do vetor de pesos descrito a seguir vale tanto para o método
do gradiente quanto para o de Newton-Raphson, portanto para o segundo será focado apenas nas
diferenças entre os dois.


\begin{algorithm}[H]
	\caption{Logistic Regression usando método do gradiente}
	\begin{algorithmic}[1]
		\REQUIRE Matriz $ X \in \mathbb{R}^{n \times m} $, 
		vetor de rótulos $t \in \{0, 1\}^n$
		\ENSURE Vetor de pesos $w \in \mathbb{R}^m$
		\STATE $iteracao \leftarrow 0$
		\STATE $w \leftarrow 0$
		\WHILE{ $|E(w)^{ (iteracao) } - E(w)^{ (iteracao - 1) } | \ge \epsilon$ \AND
		$iteracao < maxIteracoes$ } \label{lst:line:condition}
			\STATE $y \leftarrow (\sigma(w^Tx_1), \sigma(w^Tx_2), \ldots, \sigma(w^Tx_n))^T$
			\STATE $\nabla E(w) \leftarrow X^T(y - t)$
			\STATE $w \leftarrow w - \alpha \nabla E(w)$
			\STATE $E(w)^{ (iteracao) } \leftarrow 
			- \sum_{i = 1}^{n} \{ t_nln(y_n) + (1 - t_n) ln(1 - y_n) \}$
			\STATE $iteracao \leftarrow iteracao + 1$
		\ENDWHILE
	\end{algorithmic}
\end{algorithm}

Importante notar que em~\ref{lst:line:condition} tem-se duas condições de paradas do algoritmo que
são o número de iterações e a diferença da diminuição da função objetivo for menor do que
um dado $\epsilon$. Tais condições são chamadas de condições de convergência e nos garantem
que chegamos a um valor suficientemente próximo do ótimo, uma vez que atingir este valor
pode exigir um número muito alto de iterações, o que traz um custo computacional.
 Na implementação do algoritmo, escolheu-se um valores
padrão para $\epsilon$ e $maxIteracoes$ como $10^{-4}$ e $200$ respectivamente.

A quantidade de iterações necessárias para a convergência é influenciada fortemente pela
escolha de $\alpha$. Um valor pequeno para $\alpha$ acarretaria em muitas iterações até
a convergência ao passo que um valor muito grande pode fazer com que se pare muito longe
do valor ótimo.


\subsection{Método de Newton-Raphson}
\label{subsec:newton-raphson}

Vimos em \ref{subsec:grad_descent} que o método do gradiente apesar de implementação
simples pode levar muito tempo para resolver o problema.

O método de Newton-Raphson acaba convergindo mais rápido do que o método do gradiente,
contudo ao custo de uma maior complexidade devido à necessidade de calcular outros
elementos.

A atualização agora é feita seguindo a equação

\begin{center}
	\begin{equation}\label{eq:newton-raphson}
		w^{ (novo) } = w^{ (antigo) } - H^{-1} \nabla E(w)
	\end{equation}
\end{center}

Onde H é a matriz Hessiano da função erro, que é calculado usando $H = \nabla \nabla E(w)
= X^TRX$ onde R é uma matriz diagonal $n \times n$ onde as entradas da diagonal principal
valem $R_{kk} = y_k(1 - y_k)$. Substituindo os valores de H e usando
\ref{eq:gradient} em \ref{eq:newton-raphson} obtemos


\begin{equation}
\begin{split}
w^{ (novo) } & = w^{ (antigo) } - (X^T R X)^{-1} \nabla E(w) \\
	& = (X^T R X)^{-1}[(X^T R X)w^{ (antigo) } - X^T(y - t)]  
\end{split}
\end{equation}

\subsection{Extensão para o caso de várias classes}

Diversas abordagens podem ser usadas para resolver o problema multiclasse, no
caso será usado diversos discriminantes $y_k$ com $k = \{1, \ldots K\}$ com K
sendo o total de classes. Assim nosso vetor w agora é uma matriz
$W \in \mathbb{R}^{m \times k}$. 

Quanto à codificação do vetor de rótulos, 
segue-se a codificação dada em Bishop (2006)\cite{bishop2006} de $1-K$,
na codificação tem-se que $t_n \in \{0, 1\}^k$ com $t_{nk} = 1$ se o elemento
n pertencer à classe k e 0 nas demais entradas. 

Quanto a função de probabilidade que desejamos estimar, utiliza-se a função
\textit{softmax} que é dada pela equação:

\begin{center}
	\begin{equation}
		P(C^k | x_n) = y_{nk} = \frac{exp(w_k^Tx_n)}{\sum_j exp(w_j^Tx_n)} 
	\end{equation}
\end{center}

Que nos dá verossimilhança e consequentemente a seguinte função de erro, tomada
usando o negativo do logaritmo da verossimilhança.

\begin{center}
	\begin{align*}
				P(T | w_1, \ldots, w_k) &= \prod_{i = 1}^{n} \prod_{j = 1}^{k} P(C^j | x_i)^{t_{ij}} = \prod_{i = 1}^{n} \prod_{j = 1}^{k} y_{ij}^{t_{ij}} \\
		E(W) &= - \sum_{i = 1}^{n} \sum_{j = 1}^{k} t_{ij} ln(y_{ij})	
	\end{align*}
\end{center}

Novamente nesse caso pode-se encontrar o valor de W que minimize $E(W)$ usando os
dois métodos discutidos em \ref{subsec:grad_descent} e \ref{subsec:newton-raphson},
porém agora temos que a derivada com respeito a cada $w_k^Tx$ vale:

\begin{center}
	\begin{equation}\label{eq:softmax_derivative}
		\frac{\partial y_k}{\partial (w_j^Tx)} = y_k(\mathds{1}_{k == j} - y_j)
	\end{equation}
\end{center}

Nosso valor de W pode ser interpretado tanto como uma matriz $m \times k$ como um
único vetor $1 \times mk$ onde $W = (w_1, w_2, \ldots, w_k)$.

Usando essa representação, podemos calcular o vetor gradiente onde a derivada
com respeito a cada $w_j$ é dada pela equação:

\begin{center}
	\begin{equation}
		\nabla_{w_j} E(W) = X^T(Y_j - T_j)
	\end{equation}
\end{center}

Com $Y_j$ e $T_j$ correspondendo, respectivamente, às j-ésimas colunas de Y e T.

Com o gradiente em mãos já temos o que é necessário para o método do gradiente e a
atualização seria feita da forma $W^{ (novo) } = W^{ (antigo) } - \alpha \nabla E(W)$.

Para aplicarmos o método de Newton-Raphson, seria necessário computarmos o Hessiano que
nesse caso seria uma matriz $m*k \times m*k$ com cada bloco $j, i$ contendo uma matriz
$m \times m$ calculada pela equação:

\begin{center}
	\begin{equation}
		\nabla_{w_i} \nabla_{w_j} E(W) = - \sum_{k = 1}^n y_{ki}( \mathds{1}_{i == j} - y_{kj})
		X_k^TX_k
	\end{equation}
\end{center} 

Onde $X_k$ é a k-ésima linha de X. Com essas equações em mãos nossa atualização de
W seria feita usando a fórmula $W^{ (novo) } = W^{ (antigo) } - H^{-1}\nabla E(W)$.

A classificação de um novo x é feita a partir
do cálculo de $P(C^k | x) = y_k(x), \forall k = \{1, \ldots, k\}$.

A classe de x é dada pelo k que tiver a maior probabilidade sobre os demais.


\section{Suport Vector Machine}