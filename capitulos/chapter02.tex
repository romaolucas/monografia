\chapter{Revisão Bibliográfica}

Diversos estudos recentes têm sido realizados na área de análise de sentimentos.
Medhat (2014)\citep{medhat2014}, sumariza diversos estudos feitos dividindo-os em
abordagens (usando técnicas de aprendizado de máquina, dicionário léxico ou
híbrida) e em objetivos (classificação de sentimentos, detecção de emoções etc).

Em Pak (2010)\citep{pak2010} é analisado o uso de um corpus composto de tuítes para
a realização de tarefas de análise de sentimentos e mineração de opinião. Nesse estudo
coletou-se 300000 tuítes divididos igualmente entre opiniões positivas, negativas e neutras, para
isso é descrito um método de como coletar textos com opiniões positivas e negativas procurando
por textos que contivessem emoticons comumente associados a opiniões positivas e negativas e para
textos de sem nenhuma polaridade, coletou-se textos diretamente de portais de notícias.

Além da descrição de como extrair tuítes de cada categoria, o trabalho realiza uma análise linguísticas
de quais palavras são mais presentes em textos objetivos e em subjetivos, ao exemplo de 
superlativos em textos positivos e verbos conjugados na terceira pessoa ou no passado serem 
predominantes em textos objetivos. Para a classificação dos textos é usado o método Naive Bayes e
o trabalho descreve algumas formas de melhorar o desempenho do algoritmo como o uso de bigramas,
e filtrar os n-gramas na hora de treinar o modelo além de analisar como o aumento do tamanho do 
conjunto de dados contribui para a melhora da acurácia. 

Outros estudos, assim como este trabalho, têm como foco o uso de análise de sentimentos
aplicados à política utilizando tuítes como o corpus. Em Tumasjan et. al (2010)\citep{tumasjan2010} é estudado
se é possível utilizar o Twitter para obter uma previsão para os resultados de uma eleição
tendo como base as eleições do parlamento alemão realizadas em 2009.

Para isso, usou-se mais de 100 mil tuítes contendo referências a políticos ou aos respectivos partidos
e extraiu-se os sentimentos associados aos textos usando a ferramenta LIWC (Linguistic Inquiry and
Word Count), ferramenta usada para avaliar a intensidade do texto sob uma certa perspectiva (ou 
dimensão como é chamado no texto) como emoções positivas e negativas, raiva, ansiedade, raiva entre
outros,
 e a partir da análise dessa extração foi estudado se:

\begin{enumerate}
	\item A atenção dada a cada partido / candidato no Twitter reflete os resultados das eleições, onde
	utilizando o erro médio absoluto obteve-se uma taxa de erro de apenas 1,65\%, valor próximo aos
	obtidos em outros meios de pesquisa aceitos.
	\item Verificar se a rede social pode dar insights sobre possíveis alianças partidárias pós-eleições
	e quais ideologias em comum existem entre os partidos.
\end{enumerate}

Risquandl e Petković (2013)\citep{petkovic2013} realizam experimentos tendo como cenário as
eleições presidenciais estado-unidenses de 2012 para classificar sentimentos dos usuários do
Twitter, porém diferente dos outros trabalhos anteriormente mencionados, esse estudo utiliza
técnicas de \textit{part-of-speech tagging} para realizar extração de aspecto, isto é, a quem
o tuíte se refere. A técnica em questão extrai subjetivos que podem ser consideradas	 como
tópicos de campanha e mede-se a correlação entre o termo e cada candidato usando uma medida
de associação.

Por último, utilizou-se um léxico de sentimentos para classificar a opinião relativa
a cada tópico de campanha de cada candidato e, a partir disso, elaborou-se um sumário das opiniões
acerca de cada tópico para melhor visualização das opiniões dos usuários acerca de cada tópico
relevante para cada candidato.

Bakliwal et. all (2013)\citep{bakliwal2013} realiza um estudo semelhante ao deste trabalho a classificação de opinião
tendo como base três classes (positivo, negativo ou neutro) acerca de tuítes coletados sobre
as eleições gerais da Irlanda em fevereiro de 2011. A diferença entre os trabalhos é que Bakliwal
divide as classificações das opiniões entre cada partido.

Outra diferença entre este trabalho e os demais é que neste os tuítes são analisados não só por
meio de um método de aprendizado de máquina, mas também  utilizando um léxico de sentimentos.
Importante comentar que neste artigo é discutido o método como a classificação manual foi feita:
utilizando mais de um avaliador e, caso ouvesse discordância, a opinião de um terceiro seria pedida
além de especificar mais classes para as quais os tuítes poderiam ser classificados (como tuítes
que não estão em inglês ou que não falam sobre o tópico além de tuítes de opinião mista).
