\chapter{Revisão Bibliográfica}

Diversos estudos recentes têm sido realizados na área de análise de sentimentos.
Medhat (2014)\citep{medhat2014}, sumariza diversos estudos feitos dividindo-os em
abordagens (usando técnicas de aprendizado de máquina, dicionário léxico ou
híbrida) e em objetivos (classificação de sentimentos, detecção de emoções etc).

Em Pak (2010)\citep{pak2010} é analisado o uso de um corpus compostos de tuítes para
a realização de tarefas de análise de sentimentos e mineração de opinião. Nesse estudo
é construído um classificador de sentimentos para as opiniões associadas aos tuítes coletados
e é discutido desafios encontrados na hora de desenvolver um classificador para o corpus.

Outros estudos, assim como este trabalho, têm como foco o uso de análise de sentimentos
aplicados à política utilizando tuítes como o corpus. Em Tumasjan et. al (2010)\citep{tumasjan2010} é estudado
se é possível utilizar o Twitter para obter uma previsão para os resultados de uma eleição
tendo como base as eleições do parlamento alemão realizadas em 2009.

Risquandl e Petković (2013)\citep{petkovic2013} realizam experimentos tendo como cenário as
eleições presidenciais estadounidenses de 2012 para classificar sentimentos dos usuários do
twitter, porém diferente dos outros trabalhos anteriormente mencionados, esse estudo utiliza
técnicas de \textit{part-of-speech tagging} para identificar sobre quem os tuítes se referem e, a
partir disso, classifica o sentimento baseado a aspectos dessa entidade, no caso pautas eleitorais
que foram fortemente discutidas como casamento de pessoas do mesmo gênero, teaparty etc.

Bakliwal et. all (2013)\citep{bakliwal2013} realiza um estudo semelhante ao deste trabalho a classificação de opinião
tendo como base três classes (positivo, negativo ou neutro) a cerca de tuítes coletados sobre
as eleições gerais da Irlanda em fevereiro de 2011. A diferença entre os trabalhos é que Bakliwal
divide as classificações das opiniões entre cada partido.


