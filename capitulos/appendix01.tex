\chapter{Configurando e instalando o CLAM}
\label{chap:clam}

Neste apêndice, será ensinado como configurar e rodar o CLAM não só na sua máquina
local, mas também disponibilzá-lo na internet utilizando o heroku para hospedagem.
É interessante possuir uma versão do CLAM rodando localmente não só para verificar
se as configurações atendem à suas necessidades, mas também para realizar testes locais
de qualquer modificação que se deseja fazer.

\section{Breve Introdução ao Heroku}

Heroku é um serviço de nuvem \textit{platform as a service} (PaaS) -  uma categoria de serviços de
nuvem que permite que usuários desenvolvam, administrem e rodem aplicações web sem a complexidade
da construção da infraestrutura associada com o deploy da aplicação -  que dá suporte a diversas
linguagens de programação, dentre elas, Python.

Com o Heroku torna-se fácil escolher quais aplicações e tarefas devem ser executadas em produção e
provem uma vasta documentação para auxiliar a construir uma aplicação nele, portanto sendo ideal
para este tutorial.

\section{Breve Introdução ao Django}

Django é um framework web open source para Python que segue o padrão arquitetural model-view-template
(MVT) que tem como objetivo facilitar a criação de aplicações complexas para a web. Para isso, é 
pregado fortemente os princípios de reusabilidade e "plugabilidade" de componentes, desenvolvimento
rápido e o princípio DRY de engenharia de software.

O conjunto principal do Django traz consigo um \textit{object-relational mapper} (ORM) para mediar
entre os modelos de dados (representados como classes em Python) e uma base de dados relacional 
(Model), um sistema para processar requisições HTTP com um sistema de templates web (View) e um
despachador de URLs que utiliza expressões regulares (Controller). Além disso possui integrado
um admin onde é possível realizar a criação, procura, modificação e deleção de dados e este mesmo
admin pode ter suas funcionalidades estendidas, caso necessário.

\section{Preparando o ambiente local}

Antes de mais nada, será preciso instalar os seguintes programas:

\begin{itemize}
	\item PostgreSQL - banco de dados usado para a aplicação. Para instalar no Linux
	use o comando: \texttt{sudo apt-get install postgresql postgresql-contrib}
	\item Python-Pip - ferramenta para realizar o download de bibliotecas em python.
	Pode ser instalado usando o comando \texttt{sudo apt-get install python-pip}.
	\item Virtualenv - ferramenta utilizada para criar um ambiente isolado para
	o funcionamento de nossa aplicação, assim as instalações de pacotes e bibliotecas
	não irão afetar o funcionamento das demais bibliotecas do computador. Pode ser
	instalado usando o comando \texttt{sudo pip install virtualenv}.
	\item Heroku CLI - ferramenta necessária para utilizar as funcionalidades do heroku
	como fazer o deploy da aplicação, executar comando no servidor etc. Pode ser instalado
	pelo link: \url{https://devcenter.heroku.com/articles/getting-started-with-python\#set-up}
\end{itemize}

\subsection{Clonando o repositório e instalando dependências}

Uma vez instalado todos os programas necessários, é necessário clonar o repositório do CLAM.
Para isso execute o comando: 

\texttt{git clone https://github.com/romaolucas/manual-classifier-helper}.

Com o repositório clonado, vamos criar um ambiente para instalarmos todas as bibliotecas necessárias
para subir o servidor. Um ambiente é criado usando o comando \texttt{virtualenv <nome_do_ambiente>}, 
uma pasta com o nome fornecido será criado e, para ativar o ambiente, basta usar o comando 
\texttt{source <nome_do_ambiente>/bin/activate} e desativando usando o comando \texttt{deactivate}.

Agora acesse o diretório que contém o CLAM, caso não tenha renomeado, se chama \texttt{manual-classifier-helper}.
No diretório iremos instalar as dependências utilizando o comando \texttt{pip install -r requirements.txt}.

\subsection{Cadastro no Heroku}

Outro passo importante a se fazer é cadastrar-se no Heroku e criar uma aplicação em python. O cadastro
pode ser feito no link \url{https://signup.heroku.com/}. Depois de se cadastrar e confirmar o e-mail,
você será levado ao dashboard do heroku onde é possível criar um novo app. Dê um nome para a sua 
aplicação e siga os passos seguintes para criar sua aplicação.

Uma vez com a aplicação criada, iremos fazer o login na heroku CLI. Execute o comando \texttt{heroku login} para fazer o login na sua máquina local.

Após o login acessa o diretório onde o clam se encontra que, caso não tenha renomeado, se chama
\texttt{manual-classifier-helper}, no diretório iremos indicar ao git do heroku qual é o repositório
remoto onde nossa aplicação se encontra usando o comando: \texttt{heroku git:remote -a <nome_do_app>}.

\subsection{Criando banco de dados e usuário local}
\label{subsec:database}

Nesta parte, iremos ensinar como criar um usuário para o django se conectar ao Postgres.

\begin{enumerate}
	\item Acesse o Postgres com o comando: \texttt{sudo -U postgres psql}.
	\item Crie o usuário 'mc_user' com o comando: \texttt{CREATE USER mc_user WITH CREATEDB CREATEUSER PASSWORD 'cl4ss1f1c4r_d4d0\$_eh_muito_chato'}. Note que aqui definimos nome e senha de usuário conforme
	definidos no arquivo \texttt{mchelper/settings.py}. Caso queira criar outro usuário ou outra senha
	para utilizar o banco de dados, basta mudar as linhas 82 e 83 do arquivo settings.py.
	\item Crie a base de dados manual_classifier com o comando: \texttt{CREATE DATABASE manual_classifier}.
	Assim como para usuário e senha, caso queira usar outro nome de base de dados, basta
	modificar a linha 81 do arquivo settings.py.
\end{enumerate}

Criados usuário e base de dados, iremos agora pedir ao Django para que ele crie as tabelas e
aplique as migrações feitas às tabelas. Para isso, rode o comando \texttt{python manage.py migrate}.
Feito isso todas as tabelas estarão criadas e a nossa disposição para usarmos.

\section{Rodando o CLAM localmente}

Seguindo passo a passo a seção anterior, você já consegue rodar perfeitamente o clam na sua máquina
local. Para inicializar o servidor, basta fazer: \texttt{heroku local web}. Feito isso você conseguirá
acessar o clam localmente no endereço \texttt{localhost:5000}.

Para utilizar o admin, será necessário criar um superusuário, isso pode ser feito com o comando
\texttt{python manage.py createsuperuser} onde você será instruído a selecionar um nome de usuário e 
senha. Feito isso você conseguirá acessar todas as funcionalidades do CLAM.

\section{Fazendo deploy do CLAM no Heroku}

Antes de realizar o deploy da aplicação, será necessário indicar para o Django que o host
'sua_aplicação.herokuapp.com' tem a autorização para ser um host HTTP da nossa aplicação.
Isso é feito modificando a linha 28 do arquivo mchelper/settings.py trocando o 
meu-app-teste.herokuapp.com pelo nome correto de sua aplicação.

Feito isso, basta rodar os comandos:

\begin{center}
	\begin{verbatim}
		git add .
		git commit -m"<mensagem_de_commit>"
		git push heroku master #fara o deploy da aplicação
	\end{verbatim}
\end{center}

Pronto! Agora sua aplicação já está deployada. Note que agora precisaremos criar as tabelas
e executar as migrações assim como fizemos localmente além de criar um superusuário rodando
os mesmos comandos usados na subseção \ref{subsec:database}. Com isso, sua aplicação já estará
funcionando corretamente. 