\chapter*{Resumo}

Com a grande expansão da internet, o uso das redes sociais tem se
tornado cada vez mais ativo e, com o passar do tempo, tornou-se um
lugar onde os usuários relatam não só informações acerca do cotidiano,
mas também para opinar sobre temas como política, esportes, etc. O uso
 das redes sociais também causa maior mobilização dos usuários no
cenário político através de petições online ou organização de
manifestações.

Este trabalho tem como objetivo explorar a rede de usuários do Twitter
para analisar os sentimentos dos seus usuários acerca do cenário
político brasileiro atual utilizando técnicas de processamento de
linguagem natural em conjunto de técnicas de aprendizado de máquina.

Para isso coletou-se tuítes sobre grandes decisões atuais como: (1) a
Reforma da Previdência (PEC287); (2) a proposta de Lei da
Terceirização; (3) a proposta de emenda constitucional sobre o Teto
dos Gastos Públicos (PEC55); além de tuítes sobre políticos em grande
visibilidade no cenário político atual.

Os tuítes coletados foram classificados manualmente através de uma
ferramenta desenvolvida neste TCC (um site na Internet) onde era
possível realizar a tarefa de classificação de forma menos árdua e
permitir a ajuda de outras pessoas.

A preparação dos dados foi feita através de técnicas de processamento
de linguagem natural para extrair informações subjetivas dos tuítes e
classificou-se as opiniões associadas a eles utilizando técnicas de
aprendizado de máquina.

Induzidos os classificadores, analisou-se métricas de desempenho de
cada um como Acurácia, Precisão e Revocação para diferentes abordagens
de extrair as informações subjetivas do texto.
Como resultado, obteve-se uma acurácia média de 71\%.

