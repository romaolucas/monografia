\chapter*{Resumo}
\addcontentsline{toc}{chapter}{Resumo}

Com a grande expansão da internet, o uso das redes sociais tem se tornado cada vez
mais ativo e, com o passar do tempo, tornou-se um lugar onde os usuários relatam não
só informações acerca do cotidiano, mas também para opinar sobre temas como política,
esportes, etc. O uso maior das redes sociais também causa maior mobilização dos usuários
no cenário político através de petições online ou organização de manifestações.

Este trabalho tem como objetivo estudar as opiniões dos usuários do Twitter acerca
do cenário político brasileiro atual utilizando técnicas de processamento de linguagem
natural em conjunto com técnicas de aprendizado de máquina. 

Para isso coletou-se tuítes sobre grandes decisões
atuais como a reforma da previdência (PEC287), a proposta de lei da terceirização e a
proposta de emenda constitucional sobre o teto dos gastos públicos (PEC55) bem como sobre
políticos em grande visibilidade no cenário político atual. Para a classificação manual
dos tuítes, desenvolveu-se um site onde era possível realizar a tarefa de forma menos
árdua e permitir a ajuda de outras pessoas.
Aplicou-se técnicas de processamento de linguagem natural para
extrair informações subjetivas dos tuítes e classificou-se as opiniões associadas a eles
utilizando técnicas de aprendizado de máquina.
Uma vez desenvolvido os classificadores, analisou-se métricas de desempenho de cada um como
acurácia, precisão e revocação para diferentes abordagens de extrair as informações subjetivas
do texto.

Ao fim de uma primeira ronda de experimentos, percebeu-se que a baixa quantidade de tuítes com
opinião positiva atrapalhava o desempenho geral dos algoritmos e, portanto decidiu executar mais
uma ronda sem eles classificando então os tuítes apenas em negativo e neutro. Ao remover estes tuítes
obteve-se uma melhoria nos classificadores obtendo resultados médios acima de 67\% em todas as
métricas e um classificador que pode ser usado com bom desempenho, mesmo com os dados 
\textit{in natura}.

